% Created 2016-01-29 Fri 14:43
\documentclass[11pt]{article}
\usepackage[utf8]{inputenc}
\usepackage[T1]{fontenc}
\usepackage{fixltx2e}
\usepackage{graphicx}
\usepackage{grffile}
\usepackage{longtable}
\usepackage{wrapfig}
\usepackage{rotating}
\usepackage[normalem]{ulem}
\usepackage{amsmath}
\usepackage{textcomp}
\usepackage{amssymb}
\usepackage{capt-of}
\usepackage{hyperref}
\author{Uwe Brauer}
\date{\today}
\title{}
\hypersetup{
 pdfauthor={Uwe Brauer},
 pdftitle={},
 pdfkeywords={},
 pdfsubject={},
 pdfcreator={Emacs 25.1.50.1 (Org mode 8.3.1)}, 
 pdflang={English}}
\begin{document}

\tableofcontents

Test to see whether we can push

That seems to work %%% UB:28.02.2022:

Now it is a private one!

Let us see whether we can push to a private one!


\section{Collaboration via the Net, Latex files.}
\label{sec:orgheadline1}

For decades software developers have been using version control
programs and servers which hosted these programs. Scientists still are
mostly relying on email and so do not benefit from those new tools.
These notes are intended to discuss some alternatives.

\section{Useful Latex packages}
\label{sec:orgheadline2}


The following style files  are very handy and should be used
independently of a version control system or sharing latex files via a
server. 


\begin{itemize}
\item subfiles: Idea you want to split up the document into master
and children, such that different authors can work
independently on different children (sections) without much
hassle. It is an extension of input or include with one
important difference you can run the children files
individually. See  \url{https://en.wikibooks.org/wiki/LaTeX/Modular_Documents}

\item rcsinfo: allows to insert a header or footer which displays
author, date and version of the document.

\item todonotes, this is a generalisation of marginpar, that is it
adds (numbered) notes at the margins, but moreover has  the
following extensions. A list of notes at the beginning. A line
which is drawn from the note to the text in the document,
something Openoffice and MS word offer for a long time. Here is
an example of code.
\end{itemize}

\begin{verbatim}
\usepackage[colorinlistoftodos, textwidth=4cm, shadow]{todonotes}

\definecolor{GreenYellow}{rgb}{0.678431,1.000000,0.184314}
\newcounter{todocounter}
\newcommand{\todonum}[2][]
{\stepcounter{todocounter}\todo[#1]{\thetodocounter: #2}}

\newcounter{ubcomment}

\newcommand{\ubcomment}[2][]{%
\refstepcounter{ubcomment}%
{%
\todo[linecolor=black,backgroundcolor={green!40!},size=\footnotesize]{%
\textbf{Fixme: UB [\uppercase{#1}\theubcomment]:}~#2}%
}}

\newcommand{\ubcommentinline}[2][]{%
\refstepcounter{ubcomment}%
{%
\todo[linecolor=black,inline,backgroundcolor={green!40!},size=\footnotesize]{%
\textbf{Fixme: UB [\uppercase{#1}\theubcomment]:}~#2}%
}}
\end{verbatim}


\section{Useful external programs.}
\label{sec:orgheadline3}

Latex diff is very useful for comparing latex documents and generating
PDF (dvi) files which  display the changes and deletion. It is also
compatible with revision control systems, such as RCS, git and
mercurial. See for example:
\url{https://www.sharelatex.com/blog/2013/02/16/using-latexdiff-for-marking-changes-to-tex-documents.html}

Or 

\url{https://github.com/ftilmann/latexdiff/blob/master/README}

This way a PDF file looks a lot as a Office document with Track
Version on.

\section{Server(client): Dropbox etc}
\label{sec:orgheadline4}

Instead of sending documents by email one could use the cloud,
services such as dropbox. However there is a danger to overwrite an
already modified and that is why this should be best combined with
some version control system.



\section{Version control}
\label{sec:orgheadline14}

The benefit from version control is clear, one controls the history of
changes and can roll back and forward between versions. So no need to
keep a bunch of files, with say a suffix which corresponds to the
relevant start. This comes in handy when using programs such as
latexdiff. However a word of warning: in order to meaningfully compare
or visit older version, one needs to write meaningful ChangLogs which
in practice requires some discipline.
\subsection{RCS}
\label{sec:orgheadline5}
Historically (and still in use) the first was/is RCS, which is well
suited for one file. 
\subsection{CVS/Subversion}
\label{sec:orgheadline6}

\subsection{Git/Hg}
\label{sec:orgheadline7}

However if more files are involved git or
mercurial (hg) are the candidates of choices. Mercurial is a bit
easier to use, runs on more platforms (git runs natively only on
linux/unix), and has a nice graphical interface.

\subsection{Basic problem for \LaTeX{}+Version control.}
\label{sec:orgheadline8}

The programs which indicates the differences in various version are
based on changes in lines not in words (since wordwise diffs are very
resource consuming). Latexdiff is an entirely different sort of
program. Now unfortunately different latex editors break lines
differently and therefore indicate changes where there might be none.
Ideally everybody would use the same editor (Emacs) and the problem
would not exist. One idea to circumvent that problem would be
\emph{fmtlatex} which formats and indents latex documents nicely. Another
option is to use so called \emph{branches} in the version control system.

\subsection{The server (bitbucket)}
\label{sec:orgheadline9}

It is recommendable to use a server, where one can \emph{push} its changes
and \emph{pull} them if needed. There are a couple of good free ones. I
recommend \emph{bitbucket} because it works with mercurial. One can get a
free account and set up a private \emph{repositorio}. There business model
is that it allows 5 collaborator's for a private repositorio. If
groups which of collaborators which exceed this number have to pay. 

\subsection{Setting up an account on bitbucket.}
\label{sec:orgheadline10}

So the first step is to open an account on \emph{bitbucket}. I have one,
\textbf{kalthad} is the user name.

\subsection{Setting it up just in case.}
\label{sec:orgheadline11}

It is best to have a file in its home directory called .hgrc which
contains


\begin{verbatim}
# example config (see "hg help config" for more info)

[ui]
# name and email, e.g.
# username = Jane Doe <jdoe@example.com>
# for xemacs commits
# username = Uwe Brauer <oub@xemacs.org>
username = Uwe Brauer <oub@mat.ucm.es>
[trusted]
users = oub, sec076

[phases]
#publish = False

[extensions]
rebase =
shelve =
largefiles =
purge =
histedit = 
progress =


color =

hgext.convert=
keyword = 
mq = 
[keyword]
**.tex =
#demo.txt = /home/oub/tmp2/eulerpoissonstatic/section-0001.tex
[keywordmaps]
Author = {author|user}
Date = {date|utcdate}
Header = {root}/{file},v {node|short} {date|utcdate} {author|user}
Id = {file|basename},v [Br:{branches}]{latesttag}[Hg:{rev}] {date|utcdate} {author|user} Exp {author|user}
RCSFile = {file|basename},v
RCSfile = {file|basename},v
Revision = {node|short}
Source = {root}/{file},v




[paths]
default = https://kalthad@bitbucket.org/kalthad/eulerpoissonstatic


[hostfingerprints]
bitbucket.org = 46:de:34:e7:9b:18:cd:7f:ae:fd:8b:e3:bc:f4:1a:5e:38:d7:ac:24


[alias]
lg = log --template "{label('custom.rev', rev)}\t{label('custom.phase',phase)}\t{label('custom.tag',tags)}\t{desc|firstline} {label('custom.age', date|age)} {label('custom.user', author|user)}\n"



[color]
mode = ansi
custom.rev = yellow
custom.phase = bold
custom.user = cyan
custom.age = bold
custom.tag = bold yellow
status.modified = magenta bold
status.added = green bold
status.removed = red bold
status.deleted = cyan bold
status.unknown = blue bold
status.ignored = black bold


qseries.applied = blue bold underline
qseries.unapplied = black bold
qseries.missing = red bold
diff.diffline = bold
diff.extended = cyan bold
diff.file_a = red bold
diff.file_b = green bold
diff.hunk = magenta
diff.deleted = red
diff.inserted = green
diff.changed = white
diff.trailingwhitespace = bold red_background


[hooks]
changegroup.diffstat=hg diff --stat -r "p1($HG_NODE)" -r "max(descendants($HG_NODE))"
outgoing=hg diff --stat -r "p1($HG_NODE)" -r "max(descendants($HG_NODE))"
post-tag.kw = $HG kwexpand
\end{verbatim}

\subsection{Working flow.}
\label{sec:orgheadline12}

\begin{itemize}
\item user sign up in bitbucket.
\item User install mercurial mercurial,
\item and tortoisehg, the nice graphical interface.
\item a repo is generated with a easy to remember name, say
RelGlatzeNL, so the complete path is \url{https://bitbucket.org/kalthad/relglatzenl}.
\item an invitation to collaborators is sent.
\item collaborators clone the repo to their local machine. Either
using the command line, or tortoisehg. Command line command
is \emph{hg clone \url{https://bitbucket.org/kalthad/relglatzenl}.}
\item 
\end{itemize}



\subsection{Sofware For Mac users}
\label{sec:orgheadline13}
\begin{itemize}
\item mercurial:   \url{https://www.mercurial-scm.org/downloads}
\item tortoisehg: \url{http://tortoisehg.bitbucket.org/}
\end{itemize}
\end{document}

\textit{Oh, an empty article!} 

You can get started by \textbf{double clicking} this text block and begin editing. You can also click the \textbf{Insert} button below to add new block elements. Or you can \textbf{drag and drop an image} right onto this text. Happy writing!